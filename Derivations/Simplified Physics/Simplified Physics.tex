% This is the preamble

\documentclass[a4paper]{article}
\usepackage[margin=1in]{geometry}

\usepackage{fancyhdr}
\pagestyle{fancy}

\lhead{Richard Arkusinski and David Egolf}
%\chead{Digital Control}
\rhead{Simplified Physics}

\renewcommand{\headrulewidth}{0.4pt}
\renewcommand{\footrulewidth}{0.4pt}

\usepackage[utf8x]{inputenc}
\usepackage{amsmath,amscd} % For math formatting, commutative diagrams
\usepackage{graphicx}
\usepackage{hyperref} 
\usepackage{xcolor}
\hypersetup{
    colorlinks,
    linkcolor={red!50!black},
    citecolor={blue!50!black},
    urlcolor={blue!80!black}
}
\usepackage{graphicx}
\graphicspath{ {images9/} }
\usepackage{tcolorbox}
\usepackage{commath}
\usepackage{gensymb}
\usepackage{amsfonts}
\usepackage{amssymb}

\usepackage{listings}

\begin{document}
In deriving the state space equation for the motor control, we made a couple assumptions:\begin{itemize}
\item We assume that $\dot{\theta}$ can change really fast in response to a change in $\theta_{des}$. This assumption is plausible because the motor used is pretty powerful relative to the weight of the cardboard beam and ping-pong ball.
\item We assume that nothing except the motor changes $\theta$ or $\dot{\theta}$. This is a reasonable assumption because the weight of the ping-pong ball relative to the strength of the motor is very small.
\end{itemize}
We want to create the equations of motion for the full system under these simplifying assumptions.
\\\\
\textbf{Our strategy will be:}
\begin{enumerate}
\item Get the torque input in terms of $\theta_{des}$, the desired angle of the shaft
\item Get the unsimplified equations for the system
\item Assume the mass of the beam and the torque due to the ball are very small
\item Determine simplified equations for the system
\end{enumerate}
\section*{Relate $\theta_{des}$ and Input Torque}
Here is our strategy:
\begin{enumerate}
\item Neglect the mass of the ball for simplification
\item Relate torque to angular acceleration
\item Relate angular acceleration to angular velocity by differentiating relationship between angular velocity and angular position
\item Relate angular acceleration to angular position and reference angle using undifferentiated motor control equation
\item Relate torque to angular position and reference angle
\end{enumerate}

\subsection*{Neglect Mass of Ball}
If we neglect the (very small) mass of the ping-pong ball, then the angular inertia of the beam-ball system becomes constant. This simplifies everything.

\subsection*{Torque $\rightarrow$ Angular Acceleration}
The angular analogue to $F = ma$ is:
\begin{align*}
\tau = I \ddot{\theta}
\end{align*}
where $\tau$ is the net torque on the shaft, $I$ is the angular inertia of the beam (recall we neglect the mass of the ball), and $\ddot{\theta}$ is the angular acceleration of the beam.
\subsection*{Angular Acceleration $\rightarrow$ Angular Velocity}
In the document ``Proportional Motor Thoughts", we found the following to be true (under the assumptions listed at the top of this file):
\begin{align*}
\dot{\theta} = -k_p \cdot \theta + k_p \cdot \theta_{des}
\end{align*}
where $k_p$ is a positive constant of proportionality.
\\\\
Differentiating this result, we find:
\begin{align*}
\ddot{\theta} = -k_p \cdot \dot{\theta}
\end{align*}
if we assume that $\dot{\theta}$ doesn't change suddenly over this interval, and that $\theta_{des}$ is held constant over this interval. In fact, these two assumptions are really just one assumption, since $\ddot{\theta}$ will always exist unless  $\theta_{des}$ is not held constant. 

\subsection*{Angular Acceleration $\rightarrow$ Angular Position}
Combining the two equations from above (valid when $\theta_{des}$ is constant)
\begin{align*}
\dot{\theta} &= -k_p \cdot \theta + k_p \cdot \theta_{des} \\
\ddot{\theta} &= -k_p \cdot \dot{\theta}
\end{align*}
we find
\begin{align*}
\ddot{\theta} = -k_p  \cdot (-k_p \cdot \theta + k_p \cdot \theta_{des}) = k_p^2  \cdot  \theta -  k_p^2 \cdot \theta_{des} = k_p^2 \cdot (\theta - \theta_{des})
\end{align*}

\subsection*{Torque $\rightarrow$ Angular Position}
Summarizing all our work so far (valid when $\theta_{des}$ is constant):
\begin{align*}
\tau &= I \ddot{\theta} \\
\ddot{\theta} &= k_p^2 \cdot (\theta - \theta_{des})
\end{align*}
Combining these equations:
\begin{align*}
\tau &= I \cdot k_p^2 \cdot (\theta - \theta_{des})
\end{align*}
which relates the desired variables.

\section*{Unsimplified Equations of System}
In the document ``Ball on Ramp Derivation" we found the following Lagrangian equations:
\begin{align*}
\left(\frac{J_{ball}}{r_{ball}^2} + m_{ball} \right) \cdot \ddot{p} - m_{ball} \cdot p \cdot \dot{\theta}^2 + m_{ball} \cdot g \cdot \sin(\theta) = 0 \\
\left(p^2\cdot m_{ball}+ J_{ramp} \right) \cdot \ddot{\theta} + 2 \cdot p \cdot \dot{p} \cdot m_{ball} \cdot \dot{\theta} + m_{ball} \cdot g \cdot p \cdot \cos(\theta) = \tau
\end{align*}
The next step is to plug in our expressions for $\tau$, $\dot{\theta}$ and $\ddot{\theta}$ into these equations, in order to capture our assumptions about how the motor control works. 




\end{document}
